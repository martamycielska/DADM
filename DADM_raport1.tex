\documentclass[12]{article}
\usepackage{amsmath}                                            % pakiet matematyczny AMS Math
\usepackage{polski}
\usepackage[utf8]{inputenc}
\usepackage{url}
\usepackage{geometry}


\title{Dedykowane Algorytmy Diagnostyki Medycznej\\ \small{Projekt oprogramowania MRI}}
\author{Marta Mycielska}
\setlength{\parskip}{1.5ex}
\begin{document}
	
	\maketitle
	\newpage
	\tableofcontents
	\newpage
	%\newgeometry{tmargin=3cm, bmargin=3cm, lmargin=3cm, rmargin=3cm} 	
	\section{Organizacja projektu}
	\subsection{System kontroli wersji}
	\paragraph{}
	Pierwszym etapem w zarządzeniu organizacją projektu, było zaplanowanie, jak programiści będą ze sobą współpracować i dodawać kod do aplikacji. Dlatego też należało starannie wybrać odpowiedni system kontroli wersji. Jest to oprogramowanie, które umożliwia wspólną pracę nad aplikacją, kontrolowanie zmian w kodzie i daje pewność, że kod jest dodawany i aktualizowany poprawnie, bez naruszenia kodu innych osób współpracujących nad aplikacją. Dostępny wybór systemu jest bardzo szeroki, jednak ostatecznie zdecydowano się na system kontroli wersji Git.
	\paragraph{}
	Git wybrano ze względu na szereg zalet, którymi się charakteryzuje. Jest to system rozproszony, a więc każda osoba odpowiedzialna za fragment kodu, może pracować na własnym komputerze, nie przeszkadzając innym. Co więcej, programista może pracować bez podłączenia do sieci (off-line), posiadając jedynie własną, lokalną kopię repozytorium. Dzięki temu każdy może pracować w dogodnym czasie, a kod dołączać do aplikacji dopiero, gdy jest uważa fragment za gotowy.
	\paragraph{}
	Ponadto, Git jest systemem darmowym, łatwym do nauczenia i łatwym w obsłudze. Są to kolejne zalety przemawiające na jego korzyść. Programista może skupić się na opracowaniu algorytmu, a nie na potencjalnych problemach z systemem kontroli wersji.

	\subsection{Repozytorium}
	\paragraph{}
	Z uwagi na fakt, iż system kontroli wersji Git jest wykorzystywany przez GitHuba, to właśnie na tym serwisie zdecydowano się założyć repozytorium. Korzystanie z repozytorium na GitHubie jest darmowe (jeżeli jest to repozytorium publiczne) i daje możliwości bugtrackera oraz wiki. W momencie, gdy Project Manager doda użytkowników współpracujących nad projektem, tylko te osoby mogą dodawać kod do repozytorium, łącząc się z nim zdalnie z własnego komputera.
	\paragraph{}
	Link do repozytorium: 
	\url{https://github.com/martamycielska/DADM}
	\subsection{Przydział zadań w zespole}
	\paragraph{}
	Zadania w zespole zostały przydzielone następująco:
	\paragraph{}
	Software Architect: Michał Trojak
	\paragraph{}
	GUI Engineer: Szymon Wójs
	\paragraph{}
	Visualization Engineer: Magdalena Kopciuch
	\paragraph{}
	Osoby odpowiedzialne za moduły przetwarzania danych:
	\paragraph{}
	MRI reconstruction: Magdalena Hetmańska
	\paragraph{}
	Intensity inhomogeneity correction: Adrianna Gurgacz
	\paragraph{}
	Non-stationary noise estimation: Katarzyna Zawada
	\paragraph{}
	Non-stationary noise filtering \#1: Klaudia Kantor
	\paragraph{}
	Non-stationary noise filtering \#2: Patryk Kuś
	\paragraph{}
	Diffusion tensor imaging: Lidia Wójtowicz
	\paragraph{}
	Skull striping: Justyna Skibińska
	\paragraph{}
	Segmentation: Agnieszka Górowska
	\paragraph{}
	Upsampling: Blanka Hasior
	\paragraph{}
	Brain 3D: Aneta Kalamaszek
	\paragraph{}
	Oblique imaging: Magdalena Połetek
	\paragraph{}
	Osoby odpowiedzialne za poszczególne moduły są zobowiązane dokonań przeglądu literaturowego dotyczącego danego zagadnienia, a następnie na tej podstawie opracować algorytm realizujący dane zagadnienia. Szczegółowy opis modułów dostępny będzie w dalszej części dokumentacji.
	Software architect jest odpowiedzialny za ustalenie wraz z programistami zależności między modułami, sporządzenie drabinki obrazującej dane zależności i skonstruowanie szkieletu aplikacji. Ponadto, konsultując się z programistami, architect wybrał środowisko programistyczne, z którego korzystają wszyscy programiści.
	\paragraph{}
	GUI engineer to osoba odpowiedzialna za opracowanie algorytmu interfejsu użytkownika oraz ustalenie, jakie moduły będą połączone z interfejsem i w jaki sposób. Konieczne jest więc ustalenie jakie kontrolki będą potrzebne, jakie informacje użytkownik będzie wprowadzał, a jakie otrzymywał.
	\paragraph{}
	Visualisation engineer ma za zadanie pomóc każdemu modułowi, którego dane wyjściowe mają być wyświetlone. Jego zadaniem jest ustalenie, jakie to będą moduły.
	
	\subsection{Trudności organizacyjne}
	\newpage
	\section{Opis oprogramowania}
	\subsection{Założenie i cel}
	\subsection{Działanie}
	\subsection{Architektura oprogramowania}
	\newpage
	\section{Specyfikacja modułów}
	\subsection{MRI reconstruction}
		\subsubsection{Wstęp teoretyczny}
		\subsubsection{Cel modułu}
		\subsubsection{Specyfikacja wejścia-wyjścia}
		\subsubsection{Analiza algorytmu}
		\newpage
	\subsection{Intensity inhomogeneity correction}
		\subsubsection{Wstęp teoretyczny}
		\subsubsection{Cel modułu}
		\subsubsection{Specyfikacja wejścia-wyjścia}
		\subsubsection{Analiza algorytmu}
		\newpage
	\subsection{Non-stationary noise estimation}
		\subsubsection{Wstęp teoretyczny}
		\subsubsection{Cel modułu}
		\subsubsection{Specyfikacja wejścia-wyjścia}
		\subsubsection{Analiza algorytmu}
		\newpage
	\subsection{Non-stationary noise filtering \#1}
		\subsubsection{Wstęp teoretyczny}
		\subsubsection{Cel modułu}
		\subsubsection{Specyfikacja wejścia-wyjścia}
		\subsubsection{Analiza algorytmu}
		\newpage
	\subsection{Non-stationary noise filtering \#2}
		\subsubsection{Wstęp teoretyczny}
		\subsubsection{Cel modułu}
		\subsubsection{Specyfikacja wejścia-wyjścia}
		\subsubsection{Analiza algorytmu}
		\newpage
	\subsection{Skull stripping}
		\subsubsection{Wstęp teoretyczny}
		\subsubsection{Cel modułu}
		\subsubsection{Specyfikacja wejścia-wyjścia}
		\subsubsection{Analiza algorytmu}
		\newpage
	\subsection{Segmentation}
		\subsubsection{Wstęp teoretyczny}
		\subsubsection{Cel modułu}
		\subsubsection{Specyfikacja wejścia-wyjścia}
		\subsubsection{Analiza algorytmu}
		\newpage
	\subsection{Diffusion tensor imaging}
		\subsubsection{Wstęp teoretyczny}
		\subsubsection{Cel modułu}
		\subsubsection{Specyfikacja wejścia-wyjścia}
		\subsubsection{Analiza algorytmu}
		\newpage
	\subsection{Brain 3D}
		\subsubsection{Wstęp teoretyczny}
		\subsubsection{Cel modułu}
		\subsubsection{Specyfikacja wejścia-wyjścia}
		\subsubsection{Analiza algorytmu}
		\newpage
	\subsection{Oblique imaging}
		\subsubsection{Wstęp teoretyczny}
		\subsubsection{Cel modułu}
		\subsubsection{Specyfikacja wejścia-wyjścia}
		\subsubsection{Analiza algorytmu}
		\newpage
	\subsection{Upsampling}
		\subsubsection{Wstęp teoretyczny}
		\subsubsection{Cel modułu}
		\subsubsection{Specyfikacja wejścia-wyjścia}
		\subsubsection{Analiza algorytmu}
		\newpage
	\section{Interfejs użytkownika}
	\newpage
	\section{Wizualizacja danych}
	\newpage
	\addcontentsline{toc}{section}{Literatura}
	\begin{thebibliography}{99}
		
	\end{thebibliography}

\end{document}

